\documentclass[10pt]{beamer}
\usetheme{Boadilla}
%\setcounter{secnumdepth}{3} % De esta forma también se numeran las subsubsections
\setcounter{secnumdepth}{2} % De esta forma no se numeran las subsubsections

\newcommand{\E}{ \mathbb{E} }
\newcommand{\R}{ \mathbb{R} }
\newcommand{\N}{ \mathbb{N} }
\newcommand{\Z}{ \mathbb{Z} }
\newcommand{\smallR}{ {\footnotesize \mathbb{R}} }

\def\eps{\varepsilon}
\def\then{\Rightarrow}
\def\then{\Leftrightarrow}
\def\tiende{\longrightarrow}

\def\befi{\begin{figure}}
\def\enfi{\end{figure}}

\def\bc{\begin{center}}
\def\ec{\end{center}}

\def\bt{\begin{tabular}}
\def\et{\end{tabular}}

\def\barr{\begin{array}}
\def\earr{\end{array}}

\def\bi{\begin{itemize}}
\def\ei{\end{itemize}}

\def\bealf{\begin{enumerate}\renewcommand{\theenumi}{\alph{enumi}}}
\def\benum{\begin{enumerate}}
\def\eenum{\end{enumerate}}

\def\bd{\begin{description}}
\def\ed{\end{description}}

\def\bq{\begin{quotation}\noindent}
\def\eq{\end{quotation}}

\def\ul#1{\underline{#1}}

%\def\prob{\mbox{Prob}}
\def\prob{\mbox{P}}


\def\www#1{{\tt #1}}

\def\refeq#1{(\ref{#1})}

% \def\subsubsubsection#1{{\sf #1}\addtocontents{toc}{#1}}



\newcommand\var{\mbox{Var}\,}
\def\cov{\mbox{Cov}}
\def\rank{\mbox{rank}}

\def\square{{\vcenter{\vbox{\hrule height.8pt \hbox{vrule width.8pt height6pt
\kern6pt \vrule width.8pt \hrule height.8pt}}}}}


\def\dem{\noindent {\bf Proof:} }
%\def\proof{\noindent {\bf Proof:} }

\newcommand{\cqd}
{
$\hbox{ }$ \hfill $\Box$
 \protect\vspace{1ex}
}

\def\pto{$\bullet$}
\def\then{\Longrightarrow}
\def\to{\rightarrow}
\def\iff{\Longleftrightarrow}
\def\sii{\Longleftrightarrow}
\def\flecha{\longrightarrow}

\def\mediana{\mbox{med}}

\def\plim{{\mbox p}\!\lim}
\def\toP{\stackrel{P}{\to}}
\def\corr{\mbox{Corr}}

\def\stackrelmeu#1#2{\stackrel{#2}{#1}}

%  Captions
\long\def\@makecaption#1#2{
   \vskip 10pt
   \setbox\@tempboxa\hbox{{\footnotesize\bf #1}: {\footnotesize\rm #2}}
   \ifdim \wd\@tempboxa >\hsize              % IF longer than one line:
      {\footnotesize\bf #1}: {\footnotesize\rm #2}\par   % THEN set as
                                             %  ordinary paragraph.
   \else                                     % ELSE  center.
      \hbox to\hsize{\hfil\box\@tempboxa\hfil}
   \fi}

%
\newcounter{numtema}
\def\programa{\section{Program} \setcounter{numtema}{0}}
\def\tema{\stepcounter{numtema}\item[{\bf Chapter \arabic{numtema}.}]}

%
\def\iid{i.i.d.r.v.}
%
\newtheorem{docu}{Documento}
\def\paper#1{\begin{docu}\label{#1} \end{docu} }

%% Para editar el documento como libro
% \newtheorem{defi}{Definición}[chapter]
% \newtheorem{lema}{Lema}[chapter]
% \newtheorem{prop}{Proposición}[chapter]
% \newtheorem{propo}[prop]{Proposición}
% \newtheorem{teorema}{Teorema}[chapter]
% \newtheorem{corol}{Corolario}[chapter]
%% Fin Para

% Para editar transparencias
\newtheorem{defi}{Definition}
\newtheorem{lema}{Lemma}
\newtheorem{prop}{Proposition}
\newtheorem{propo}[prop]{Proposition}
\newtheorem{teorema}{Theorem}
\newtheorem{corol}{Corollary}
% Fin Para

%%%%%%%%% definiciones nuevas, a partir de julio del 2000
%% Para editar el documento como libro
%\newtheorem{ejemplo}{Ejemplo}[chapter]
%% Fin Para

% Para editar transparencias
\newtheorem{ejemplo}{Example}
% Fin Para

\def\bej{\vspace*{1em}
\begin{ejemplo}\rm
\noindent\hrulefill\\}
\def\eej{\\.\hrulefill.\\
\end{ejemplo}}

\def\bejcont#1{\vspace*{1em}
\noindent\hrulefill\\ {\bf Example \ref{#1}, page \pageref{#1}. Continuation.} }
\def\eejcont{\\.\hrulefill.\\}

%\def\titulo#1{\medskip\noindent{\sf #1}\newline}
\def\titulo#1{\medskip\noindent{\large \sf #1}\newline}

\def\w{\omega}
\def\W{\Omega}

\def\A{\mathcal{A}}
\def\B{\mathcal{B}}
\def\C{\mathcal{C}}
\def\D{\mathcal{D}}
\def\cD{\mathcal{D}}
\def\F{\mathcal{F}}
%\def\ll{\mbox{{\it l$\;$}}}
\def\ll{\ell}
\def\L{\mathcal{L}}
\def\T{\mathcal{T}}
\def\S{\mathcal{S}}
\def\cI{\mathcal{I}}

\def\P{\mathbf{P}}

\def\cN{\mathcal{N}}
\def\calN{\mathcal{N}}
\def\cY{\mathcal{Y}}
\def\calY{\mathcal{Y}}
\def\cZ{\mathcal{Z}}
\def\calZ{\mathcal{Z}}
\def\cX{\mathcal{X}}
\def\calX{\mathcal{X}}
\def\X{\!\!\begin{array}[t]{c}X\\[-1.2ex]\!\!\widetilde{\;\;\;}\\[-1.7ex]\end{array}\!\!}
\def\x{\!\!\begin{array}[t]{c}x\\[-1.2ex]\widetilde{\;}\\[-1.7ex]\end{array}\!\!}

\def\Y{\!\!\begin{array}[t]{c}Y\\[-1.2ex]\!\!\widetilde{\;\;\;}\\[-1.7ex]\end{array}\!\!}
\def\y{\!\!\begin{array}[t]{c}y\\[-1.1ex]\widetilde{\;}\\[-1.9ex]\end{array}\!\!}

\def\a{\!\!\begin{array}[t]{c}a\\[-1.2ex]\widetilde{\;}\\[-1.7ex]\end{array}\!\!}
\def\vcero{\!\!\begin{array}[t]{c}0\\[-1.2ex]\widetilde{\;}\\[-1.7ex]\end{array}\!\!}

%\def\refers#1{\ \smallsize{\sc References:} #1}
\def\refers#1{\hspace*{\fill}\begin{minipage}{10cm}\raggedleft{\small
      {\sc References:} #1}\end{minipage}\medskip }

\def\med#1{\overline{#1}}

\def\si{\mbox{ si }}

\def\sumn{\sum_{i=1}^n}
\def\prodn{\prod_{i=1}^n}

\def\fa{\mbox{ for all }} %\fa: for all
\def\cs{\mbox{ almost sure}}

\def\vol{\mbox{Volume}}

\def\emv{maximum likelihood estimator}
\def\mle{maximum likelihood estimator}

\def\parcial#1#2#3{{{\partial^{#3}} #1\over \partial #2^{#3}}}
\def\deriv#1#2#3{{{d^{#3}} #1\over d#2^{#3}}}
\def\grad{\nabla}

\def\th{\theta}
\def\Th{\Theta}
\def\hth{\hat{\theta}}
\def\tth{\!\!\begin{array}[t]{c}\theta\\[-1.2ex]\widetilde{\;\;\;}\\[-1.7ex]\end{array}\!\!}
\def\htth{\hat{\!\!\begin{array}[t]{c}\theta\\[-1.2ex]\widetilde{\;\;\;}\\[-1.7ex]\end{array}\!\!}}

\def\teta{\!\!\begin{array}[t]{c}\eta\\[-1.2ex]\widetilde{\;\;\;}\\[-1.7ex]\end{array}\!\!}

\def\Bern{\mbox{Bern}}
\def\Pois{\mbox{Poisson}}

\def\cte{\mbox{cte}}

\def\umvue{UMVUE}
\def\CR{Cramér-Rao}
\def\RB{Rao-Blackwell}
\def\LS{Lehmann-Scheffé}
\def\TCL{Central Limit Theorem}
\def\CLT{Central Limit Theorem}

\def\AN{\mbox{\em AN}}
\def\ARE{\mbox{\em ARE}}

\def\test#1#2{\left\{\begin{array}{l}H_0: #1\\H_1:
      #2\end{array}\right.}

\def\deffun#1#2#3#4#5{\begin{array}{rcccl}
#1 &:& #2 & \longrightarrow & #3\\
   & & #4 & \longrightarrow & #5
\end{array}}

\def\len{\mbox{Long}}

\def\yt{\mbox{ y }} % 'y' como texto dentro de matemáticas

%%%%%%%%%%%%%%%% Control de comentarios al margen
% Con la siguiente definición son omitidos
\def\omitir#1{}
\renewcommand{\marginpar}{\omitir}
%%% Activar la linea siguiente para que salgan impresos como notas a
%%% pie de página
%\renewcommand{\marginpar}{\footnote}



%%%%% comandos hojas de problemas

%\def\seccion#1{\bigskip \pagebreak[1] \noindent {\bf #1} }
\def\seccion#1{\bigskip \item[] {\large \bf #1} }

\def\df{\text{df}}
\def\sign{\text{sign}}

\def\ba{{\bf a}}
\def\bb{{\bf b}}
\def\be{{\bf e}}
\def\ff{{\bf f}}
\def\bh{{\bf h}}
\def\bn{{\bf n}}
\def\bp{{\bf p}}
\def\br{{\bf r}}
\def\bt{{\bf t}}
\def\bv{{\bf v}}
\def\bx{{\bf x}}
\def\by{{\bf y}}
\def\bw{{\bf w}}
\def\bz{{\bf z}}
\def\bA{{\bf A}}
\def\bD{{\bf D}}
\def\bG{{\bf G}}
\def\bH{{\bf H}}
\def\bI{{\bf I}}
\def\bK{{\bf K}}
\def\bL{{\bf L}}
\def\bM{{\bf M}}
\def\bN{{\bf N}}
\def\bU{{\bf U}}
\def\bV{{\bf V}}
\def\bW{{\bf W}} 
\def\bX{{\bf X}}
\def\bY{{\bf Y}}
\def\bZ{{\bf Z}}%{{\mathbf Z}}

\def\bnu{\boldsymbol\nu}
\def\bmu{\boldsymbol\mu}
\def\balpha{\boldsymbol\alpha}
\def\bbeta{\boldsymbol\beta}
\def\blambda{\boldsymbol\lambda}
\def\bepsilon{\boldsymbol\varepsilon}
\def\btheta{\boldsymbol\theta}
\def\brho{\boldsymbol\rho}

\def\E{\mathbb{E}}
\def\one{\mathbb{I}}

\def\bzero{\mathbf 0}

\def\ave{\mbox{ave}}
\def\oob{{\sc oob}}

\def\diag{\mbox{Diag}}
\def\Trace{\mbox{Trace}}
\def\trace{\mbox{Trace}}
\def\tr{^{\mbox{\scriptsize T}}}


%% Changing the background color of a frame
\def\frameyellow#1{
	{\setbeamercolor{background canvas}{bg=yellow!20}
		\begin{frame}
			#1
		\end{frame}
	}
}
%% Changing the background color of a frame
\def\frameblue#1{
	{\setbeamercolor{background canvas}{bg=blue!10}
		\begin{frame}
			#1
		\end{frame}
	}
}



%%%%%%%%%%%%%%%%%% To include source code in slides
% include the source code between
% \begin{lstlisting}
%  ...
% \end{lstlisting}

\usepackage{listings}
\usepackage{color}

\definecolor{mygreen}{rgb}{0,0.6,0}
\definecolor{lightgray}{rgb}{0.95,0.95,0.95}
\definecolor{mygray}{rgb}{0.5,0.5,0.5}
\definecolor{mymauve}{rgb}{0.58,0,0.82}

\lstset{ 
	backgroundcolor=\color{lightgray},   % choose the background color; you must add \usepackage{color} or \usepackage{xcolor}; should come as last argument
	basicstyle=\footnotesize\ttfamily,        % the size of the fonts that are used for the code
	breakatwhitespace=false,         % sets if automatic breaks should only happen at whitespace
	breaklines=true,                 % sets automatic line breaking
	captionpos=b,                    % sets the caption-position to bottom
	commentstyle=\itshape\color{mygreen},    % comment style
	deletekeywords={...},            % if you want to delete keywords from the given language
	escapeinside={\%*}{*)},          % if you want to add LaTeX within your code
	extendedchars=true,              % lets you use non-ASCII characters; for 8-bits encodings only, does not work with UTF-8
	firstnumber=1000,                % start line enumeration with line 1000
	frame=none,	                   % adds a frame around the code
	keepspaces=true,                 % keeps spaces in text, useful for keeping indentation of code (possibly needs columns=flexible)
	keywordstyle=\bfseries\color{blue},       % keyword style
	language=R,                      % the language of the code
	morekeywords={*,...},            % if you want to add more keywords to the set
	numbers=none,                    % where to put the line-numbers; possible values are (none, left, right)
	numbersep=5pt,                   % how far the line-numbers are from the code
	numberstyle=\tiny\color{mygray}, % the style that is used for the line-numbers
	rulecolor=\color{black},         % if not set, the frame-color may be changed on line-breaks within not-black text (e.g. comments (green here))
	showspaces=false,                % show spaces everywhere adding particular underscores; it overrides 'showstringspaces'
	showstringspaces=false,          % underline spaces within strings only
	showtabs=false,                  % show tabs within strings adding particular underscores
	stepnumber=2,                    % the step between two line-numbers. If it's 1, each line will be numbered
	stringstyle=\color{mymauve},     % string literal style
	tabsize=2,	                   % sets default tabsize to 2 spaces
	title=\lstname                   % show the filename of files included with \lstinputlisting; also try caption instead of title
}


% % % % % % Curvas principales: princurv.sty
\newcommand{\al}{\alpha} 
\newcommand{\talpha}{\hat{\balpha}} 
\newcommand{\tal}{\talpha} 
%\newcommand{\x}{x^0} 
\newcommand{\SV}{\widetilde{\mbox{Var}}}
\newcommand{\SE}{\widetilde{\bmu}} 
\newcommand{\STV}{\widetilde{\phi}} 
\newcommand{\wtb}{\widetilde{\bb}} 
\newcommand{\smu}{\widetilde{\bmu}} 
\newcommand{\sphi}{\widetilde{\phi}} 
\newcommand{\prev}{\mbox{Prev}} 
\newcommand{\wh}{\widehat}
%\newcommand{\var}{\mbox{Var}} 

\newcommand{\HSPC}{{HSPC}} 
\newcommand{\GHSPC}{{GHSPC}} 
\newcommand{\GTV}{\mbox{GTV}} 
\newcommand{\GE}{\mbox{GE}} 
\def\supp{\mbox{Support}}
\def\remark{\medskip\pagebreak[1]\noindent{\bf Remark.\ }}

\def\Remp{R_{\footnotesize\rm emp}}
\def\Rreg{R_{\footnotesize\rm reg}}




% Package imports
\usepackage{amsmath,amssymb}
\usepackage{tikz}
\usepackage{pgfplots}
\usepackage{algorithm2e}
\usepackage{subcaption}

% Bibliography setup
\usepackage[authoryear,round]{natbib}
\bibliographystyle{chicago}

% Document information
\title{t-SNE: Visualizing High-Dimensional Data Through Probabilistic Embeddings}
\subtitle{Advanced Multivariate Analysis\\Master in Statistics and Data Science}
\author{Endri Raco, PhD}
\institute{Universitat Politècnica de Catalunya}
\date{October, 2025}

\begin{document}

%==============================================================================
% SLIDE 1: Title Slide
%==============================================================================
\begin{frame}
\titlepage
\end{frame}

%==============================================================================
% SLIDE 2: The Fundamental Challenge
%==============================================================================
\begin{frame}{The Fundamental Challenge of High-Dimensional Visualization}

\begin{columns}
\begin{column}{0.5\textwidth}
\textbf{The Problem We Face Daily:}
\vspace{0.3cm}

Consider the MNIST digit dataset:
\begin{itemize}
    \item Each image: 784 dimensions (28×28 pixels)
    \item 60,000 training samples
    \item 10 distinct classes
\end{itemize}

\vspace{0.5cm}
\textbf{The Core Question:}\\
\emph{How do we visualize the structure of data living in $\mathbb{R}^{784}$ on a 2D screen?}

\end{column}
\begin{column}{0.5\textwidth}
\begin{figure}
\centering
\begin{tikzpicture}[scale=0.65]
    % High-dimensional space
    \draw[thick] (0,0) circle (1.8cm);
    \node at (0,-2.3) {$\mathbb{R}^{784}$};
    
    % Points in high-D
    \foreach \i in {1,...,8} {
        \pgfmathsetmacro{\angle}{360/8*\i}
        \pgfmathsetmacro{\radius}{1.3+0.3*rand}
        \fill[blue!70] (\angle:\radius) circle (2pt);
    }
    
    % Arrow to low-D
    \draw[->,thick,red!70] (2.2,0) -- (3.8,0);
    \node[above] at (3,0.2) {\small Embedding};
    
    % Low-dimensional space
    \draw[thick] (5.8,0) circle (1.4cm);
    \node at (5.8,-2) {$\mathbb{R}^{2}$};
    
    % Points in low-D (scattered)
    \foreach \i in {1,...,5} {
        \pgfmathsetmacro{\angle}{360/5*\i}
        \pgfmathsetmacro{\radius}{0.7+0.3*rand}
        \fill[blue!30] (5.8,0) +(\angle:\radius) circle (2pt);
    }
    
    % Question mark
    \node[red,scale=1.8] at (5.8,0) {?};
\end{tikzpicture}
\end{figure}

\vspace{0.3cm}
\textbf{The Stakes:}\\
Good visualization reveals patterns.\\
Poor visualization hides structure.
\end{column}
\end{columns}

\end{frame}

%==============================================================================
% SLIDE 3: When Classical Methods Fail
%==============================================================================
\begin{frame}{When Classical Methods Fail: PCA on Complex Data}

\vspace{-0.2cm}
\begin{figure}
\centering
\begin{tikzpicture}[scale=0.9]
    % PCA Result (Left)
    \begin{scope}[shift={(0,0)}]
        \begin{axis}[
            width=5cm, height=4.5cm,
            xlabel={PC1},
            ylabel={PC2},
            title={PCA Projection},
            xmin=-3, xmax=3,
            ymin=-3, ymax=3,
            axis equal,
            grid=major,
            grid style={gray!30}
        ]
        % Overlapping clusters
        \addplot[only marks, mark=*, mark size=1pt, blue!60, opacity=0.6] 
            coordinates {(-0.5,0.2) (-0.3,0.4) (-0.6,0.1) (-0.4,0.3) (-0.2,0.5)
                        (0.3,0.1) (0.5,0.3) (0.2,0.4) (0.4,0.2) (0.6,0.5)
                        (-0.1,-0.3) (0.1,-0.2) (-0.2,-0.4) (0.2,-0.1) (0,-0.5)};
        \addplot[only marks, mark=*, mark size=1pt, red!60, opacity=0.6] 
            coordinates {(-0.4,0.1) (-0.2,0.3) (-0.3,0.2) (-0.1,0.4) (0,0.2)
                        (0.4,0.3) (0.2,0.1) (0.3,0.4) (0.5,0.2) (0.1,0.3)
                        (0,-0.2) (-0.1,-0.1) (0.1,-0.3) (-0.2,-0.2) (0.2,-0.4)};
        \addplot[only marks, mark=*, mark size=1pt, green!60, opacity=0.6] 
            coordinates {(-0.3,-0.1) (-0.1,0.1) (-0.2,0) (0,0.1) (-0.4,0)
                        (0.3,0) (0.1,0.2) (0.4,0.1) (0.2,0) (0.5,0.1)
                        (-0.1,-0.4) (0.1,-0.1) (0,-0.3) (0.2,-0.2) (-0.2,-0.5)};
        \end{axis}
    \end{scope}
    
    % t-SNE Result (Right)
    \begin{scope}[shift={(6.5,0)}]
        \begin{axis}[
            width=5cm, height=4.5cm,
            xlabel={t-SNE 1},
            ylabel={t-SNE 2},
            title={t-SNE Projection},
            xmin=-3, xmax=3,
            ymin=-3, ymax=3,
            axis equal,
            grid=major,
            grid style={gray!30}
        ]
        % Well-separated clusters
        \addplot[only marks, mark=*, mark size=1pt, blue!80] 
            coordinates {(-2.0,1.5) (-1.8,1.7) (-2.1,1.4) (-1.9,1.6) (-2.2,1.8)
                        (-2.0,1.3) (-1.7,1.5) (-2.1,1.7) (-1.9,1.4) (-2.0,1.6)};
        \addplot[only marks, mark=*, mark size=1pt, red!80] 
            coordinates {(1.5,1.8) (1.7,2.0) (1.4,1.9) (1.6,1.7) (1.8,1.9)
                        (1.5,2.1) (1.7,1.8) (1.4,2.0) (1.6,1.9) (1.8,1.7)};
        \addplot[only marks, mark=*, mark size=1pt, green!80] 
            coordinates {(0.2,-1.8) (0.4,-2.0) (0.1,-1.7) (0.3,-1.9) (0.5,-1.8)
                        (0.2,-2.1) (0.4,-1.9) (0.1,-2.0) (0.3,-1.8) (0.5,-2.0)};
        \end{axis}
    \end{scope}
\end{tikzpicture}
\end{figure}

\vspace{-0.1cm}
\textbf{Key Observation:} PCA preserves \emph{global} structure by maximizing variance.\\
\textbf{The Problem:} Local neighborhoods—where the real clustering happens—get destroyed.

\vspace{0.15cm}
\begin{alertblock}{The Core Insight}
What if, instead of preserving distances, we preserved \emph{neighborhoods}?
\end{alertblock}

\end{frame}

%==============================================================================
% SLIDE 4: The Core Idea
%==============================================================================
\begin{frame}{The Core Idea: Preserving Local Neighborhoods}

\begin{center}
\textbf{\Large The Fundamental Principle of t-SNE}
\end{center}

\vspace{0.3cm}

\begin{columns}
\begin{column}{0.5\textwidth}
\centering
\textbf{In High Dimensions ($\mathbb{R}^{d}$)}\\
\vspace{0.2cm}
\begin{tikzpicture}[scale=1.0]
    % Center point
    \fill[red] (0,0) circle (3pt);
    \node[below] at (0,-0.2) {$\mathbf{x}_i$};
    
    % Neighbors
    \fill[blue] (0.7,0.5) circle (2pt);
    \node[right] at (0.7,0.5) {\tiny $\mathbf{x}_j$};
    
    \fill[blue] (-0.6,0.4) circle (2pt);
    \fill[blue] (0.3,-0.5) circle (2pt);
    
    % Distant points
    \fill[gray!50] (2.0,1.5) circle (2pt);
    \fill[gray!50] (-1.8,1.3) circle (2pt);
    \fill[gray!50] (1.5,-1.7) circle (2pt);
    
    % Neighborhood circle
    \draw[dashed,blue!50] (0,0) circle (1cm);
    \node[blue] at (0,1.3) {\small Neighborhood};
\end{tikzpicture}

\vspace{0.15cm}
Point $\mathbf{x}_i$ has nearby neighbors\\
(small distances in $\mathbb{R}^{d}$)

\end{column}
\begin{column}{0.5\textwidth}
\centering
\textbf{In Low Dimensions ($\mathbb{R}^{2}$)}\\
\vspace{0.2cm}
\begin{tikzpicture}[scale=1.0]
    % Center point
    \fill[red] (0,0) circle (3pt);
    \node[below] at (0,-0.2) {$\mathbf{y}_i$};
    
    % Same neighbors preserved
    \fill[blue] (0.7,0.5) circle (2pt);
    \node[right] at (0.7,0.5) {\tiny $\mathbf{y}_j$};
    
    \fill[blue] (-0.6,0.4) circle (2pt);
    \fill[blue] (0.3,-0.5) circle (2pt);
    
    % Distant points remain distant
    \fill[gray!50] (2.0,1.5) circle (2pt);
    \fill[gray!50] (-1.8,1.3) circle (2pt);
    \fill[gray!50] (1.5,-1.7) circle (2pt);
    
    % Preserved neighborhood
    \draw[dashed,blue!50] (0,0) circle (1cm);
    \node[blue] at (0,1.3) {\small Preserved!};
\end{tikzpicture}

\vspace{0.15cm}
Same neighbors stay close\\
(neighborhood structure preserved)

\end{column}
\end{columns}

\vspace{0.3cm}

\begin{block}{The Strategy}
\centering
\textbf{If point $j$ is a neighbor of point $i$ in high dimensions,}\\
\textbf{then $j$ should remain a neighbor of $i$ in low dimensions.}
\end{block}

\end{frame}

%==============================================================================
% SLIDE 5: From Distance to Similarity
%==============================================================================
\begin{frame}{Building the Mathematics: From Distance to Similarity}

\textbf{Step 1: The Raw Material—Euclidean Distance}

\vspace{0.2cm}
We start with points $\mathbf{x}_i, \mathbf{x}_j \in \mathbb{R}^d$ and compute:
\begin{equation}
d_{ij} = \|\mathbf{x}_i - \mathbf{x}_j\|_2 = \sqrt{\sum_{k=1}^{d} (x_{ik} - x_{jk})^2}
\end{equation}

\vspace{0.3cm}
\textbf{The Problem:} Distance alone doesn't capture "neighborness"
\begin{itemize}
    \item Distance = 1.0 might be "close" in sparse regions
    \item Distance = 1.0 might be "far" in dense regions
    \item We need a \emph{relative} measure, not absolute
\end{itemize}

\vspace{0.3cm}
\begin{figure}
\centering
\begin{tikzpicture}[scale=0.8]
    % Sparse region
    \begin{scope}[shift={(0,0)}]
        \node[above] at (0,1.2) {\small Sparse Region};
        \fill[red] (0,0) circle (2pt);
        \fill[blue] (1.2,0) circle (2pt);
        \draw[<->,thick] (0.1,-0.3) -- (1.1,-0.3);
        \node[below] at (0.6,-0.5) {\tiny $d=1.0$};
        \node[green!70!black] at (0.6,-0.9) {\small "Close"};
    \end{scope}
    
    % Dense region
    \begin{scope}[shift={(4.5,0)}]
        \node[above] at (0,1.2) {\small Dense Region};
        \fill[red] (0,0) circle (2pt);
        \fill[gray] (0.3,0.2) circle (1.5pt);
        \fill[gray] (-0.2,0.3) circle (1.5pt);
        \fill[gray] (0.4,-0.2) circle (1.5pt);
        \fill[blue] (1.2,0) circle (2pt);
        \draw[<->,thick] (0.1,-0.3) -- (1.1,-0.3);
        \node[below] at (0.6,-0.5) {\tiny $d=1.0$};
        \node[red!70!black] at (0.6,-0.9) {\small "Far"};
    \end{scope}
\end{tikzpicture}
\end{figure}

\end{frame}

%==============================================================================
% SLIDE 6: Converting Distance to Similarity
%==============================================================================
\begin{frame}{Building the Mathematics: Converting Distance to Similarity}

\textbf{Step 2: The Gaussian Kernel—From Distance to Similarity}

\vspace{0.15cm}
We need to convert distance to a similarity score between 0 and 1:
\begin{equation}
\text{similarity}_{ij} = \exp\left(-\frac{d_{ij}^2}{2\sigma_i^2}\right)
\end{equation}

\vspace{0.2cm}
\textbf{Why the Gaussian kernel?}
\begin{itemize}
    \item Close points ($d_{ij} \approx 0$) → similarity $\approx 1$
    \item Far points ($d_{ij} \gg \sigma_i$) → similarity $\approx 0$
    \item Smooth, differentiable decay
    \item The parameter $\sigma_i$ controls the "width" of the neighborhood
\end{itemize}

\vspace{0.15cm}
\begin{figure}
\centering
\begin{tikzpicture}[scale=0.8]
    \begin{axis}[
        width=8cm, height=3.8cm,
        xlabel={Distance $d_{ij}$},
        ylabel={Similarity},
        xmin=0, xmax=4,
        ymin=0, ymax=1.1,
        legend pos=north east,
        legend style={font=\tiny},
        grid=both,
        grid style={gray!30},
        label style={font=\small},
        tick label style={font=\tiny}
    ]
    % Different sigma values
    \addplot[blue, thick, smooth, domain=0:4, samples=100] 
        {exp(-x^2/(2*0.5^2))};
    \addlegendentry{$\sigma_i = 0.5$}
    
    \addplot[red, thick, smooth, domain=0:4, samples=100] 
        {exp(-x^2/(2*1.0^2))};
    \addlegendentry{$\sigma_i = 1.0$}
    
    \addplot[green!70!black, thick, smooth, domain=0:4, samples=100] 
        {exp(-x^2/(2*1.5^2))};
    \addlegendentry{$\sigma_i = 1.5$}
    \end{axis}
\end{tikzpicture}
\end{figure}

\end{frame}

%==============================================================================
% SLIDE 7: The Bandwidth Problem
%==============================================================================
\begin{frame}{The Critical Problem: One $\sigma$ Does Not Fit All}

\begin{center}
\textbf{\large Different regions need different bandwidths!}
\end{center}

\vspace{0.3cm}

\begin{columns}
\begin{column}{0.5\textwidth}
\centering
\textbf{Dense Region}\\
\vspace{0.2cm}
\begin{tikzpicture}[scale=0.8]
    % Dense cluster
    \fill[red] (0,0) circle (3pt);
    \node[below] at (0,-0.3) {$\mathbf{x}_i$};
    
    % Many nearby points
    \foreach \angle/\dist in {30/0.4, 60/0.5, 120/0.45, 180/0.5, 240/0.4, 300/0.55} {
        \fill[blue!70] (\angle:\dist) circle (2pt);
    }
    
    % Small sigma circle
    \draw[dashed, orange, thick] (0,0) circle (0.6cm);
    \node[orange] at (0,-1.2) {Need small $\sigma_i$};
\end{tikzpicture}

\vspace{0.2cm}
Many neighbors nearby\\
→ Use \textbf{narrow} kernel

\end{column}
\begin{column}{0.5\textwidth}
\centering
\textbf{Sparse Region}\\
\vspace{0.2cm}
\begin{tikzpicture}[scale=0.8]
    % Sparse point
    \fill[red] (0,0) circle (3pt);
    \node[below] at (0,-0.3) {$\mathbf{x}_j$};
    
    % Few distant points
    \fill[blue!70] (1.2, 0.8) circle (2pt);
    \fill[blue!70] (-1.0, -0.9) circle (2pt);
    \fill[blue!70] (0.9, -1.1) circle (2pt);
    
    % Large sigma circle
    \draw[dashed, orange, thick] (0,0) circle (1.3cm);
    \node[orange] at (0,-1.8) {Need large $\sigma_j$};
\end{tikzpicture}

\vspace{0.2cm}
Few neighbors, all distant\\
→ Use \textbf{wide} kernel

\end{column}
\end{columns}

\vspace{0.3cm}

\begin{alertblock}{The Key Insight}
Each point $i$ needs its own $\sigma_i$ adapted to its local density!\\
But how do we choose these $\sigma_i$ values systematically?
\end{alertblock}

\end{frame}

%==============================================================================
% SLIDE 8: Perplexity Solution
%==============================================================================
\begin{frame}{The Solution: Perplexity—A User-Friendly Parameter}

\textbf{The Perplexity Concept:} "How many effective neighbors should each point have?"

\vspace{0.15cm}
\begin{block}{Definition of Perplexity}
\begin{equation}
\text{Perp}(P_i) = 2^{H(P_i)} = 2^{-\sum_j p_{j|i} \log_2 p_{j|i}}
\end{equation}
where $H(P_i)$ is the Shannon entropy of the distribution $P_i$.
\end{block}

\vspace{0.1cm}
\begin{exampleblock}{Shannon Entropy Connection}
\footnotesize
Entropy measures uncertainty: for uniform distribution over $k$ outcomes, $H = \log_2(k)$\\
Thus: Perp $= 2^{\log_2(k)} = k$ — the "effective" number of outcomes\\
\textit{Reference: Shannon (1948), "A Mathematical Theory of Communication"}
\end{exampleblock}

\vspace{0.1cm}
\textbf{Intuition:}
\begin{itemize}
    \setlength\itemsep{0em}
    \item Perplexity $\approx$ effective number of neighbors
    \item User sets one global perplexity (typically 5–50)
    \item Algorithm finds $\sigma_i$ for each point to match this perplexity
    \item Perplexity = 30 means "each point should have roughly 30 effective neighbors"
\end{itemize}

\vspace{0.05cm}
\begin{figure}
\centering
\begin{tikzpicture}[scale=0.6]
    % Binary search illustration
    \draw[thick,->] (0,0) -- (6,0) node[right] {\tiny $\sigma_i$};
    \draw[thick,->] (0,0) -- (0,2) node[above] {\tiny Perplexity};
    
    % Target perplexity line
    \draw[dashed, red, thick] (0,1.1) -- (6,1.1);
    \node[red, left] at (0,1.1) {\tiny Target};
    
    % Perplexity curve
    \draw[blue, thick, smooth] plot coordinates {
        (0.5,0.3) (1,0.5) (2,0.75) (3,1.0) (3.2,1.1) (4,1.3) (5,1.6) (5.8,1.85)
    };
    
    % Binary search point
    \fill[green!70!black] (3.2,1.1) circle (2.5pt);
    \node[below] at (3.2,-0.1) {\tiny $\sigma_i^*$};
    
    \node[blue] at (4.5,0.8) {\tiny Binary search};
\end{tikzpicture}
\end{figure}

\end{frame}

%==============================================================================
% SLIDE 9: Building Probability Distributions
%==============================================================================
\begin{frame}{From Similarities to Probabilities: The Full Formula}

\textbf{Step 3: Normalizing to Create Probability Distributions}

\vspace{0.2cm}
We convert similarities to conditional probabilities:
\begin{equation}
p_{j|i} = \frac{\exp\left(-\|\mathbf{x}_i - \mathbf{x}_j\|^2 / 2\sigma_i^2\right)}{\sum_{k \neq i} \exp\left(-\|\mathbf{x}_i - \mathbf{x}_k\|^2 / 2\sigma_i^2\right)}
\end{equation}

\vspace{0.2cm}
\textbf{Properties:}
\begin{itemize}
    \item $p_{j|i} \in [0,1]$ for all $i,j$
    \item $\sum_{j \neq i} p_{j|i} = 1$ (true probability distribution)
    \item $p_{i|i} = 0$ by convention (no self-similarity)
    \item $p_{j|i} \neq p_{i|j}$ in general (asymmetric)
\end{itemize}

\vspace{0.2cm}
\textbf{Interpretation:} $p_{j|i}$ = "If I pick a neighbor of point $i$ proportional to similarity, what's the probability I pick point $j$?"

\vspace{0.2cm}
\begin{figure}
\centering
\begin{tikzpicture}[scale=0.8]
    % Point i and its probability distribution
    \fill[red] (0,0) circle (3pt);
    \node[below] at (0,-0.2) {$\mathbf{x}_i$};
    
    % Neighbors with probability values
    \fill[blue] (1.0,0.3) circle (2pt);
    \node[right] at (1.0,0.3) {\tiny $p_{j|i}=0.3$};
    
    \fill[blue] (0.5,0.8) circle (2pt);
    \node[above] at (0.5,0.8) {\tiny $p_{k|i}=0.4$};
    
    \fill[blue] (-0.7,0.5) circle (2pt);
    \node[left] at (-0.7,0.5) {\tiny $p_{l|i}=0.2$};
    
    \fill[blue!40] (2.0,1.5) circle (2pt);
    \node[right] at (2.0,1.5) {\tiny $p_{m|i}=0.08$};
    
    \fill[blue!20] (-1.8,-1.3) circle (2pt);
    \node[left] at (-1.8,-1.3) {\tiny $p_{n|i}=0.02$};
    
    % Sum notation
    \node[green!60!black] at (3.5,0) {$\sum = 1.0$};
\end{tikzpicture}
\end{figure}

\end{frame}

%==============================================================================
% SLIDE 10: Low-Dimensional Mapping
%==============================================================================
\begin{frame}{Mapping to Low Dimensions: The Target Distribution}

\textbf{In the low-dimensional space ($\mathbb{R}^2$):}

\vspace{0.15cm}
We have map points $\mathbf{y}_i, \mathbf{y}_j \in \mathbb{R}^2$ and define:
\begin{equation}
q_{j|i} = \frac{\exp\left(-\|\mathbf{y}_i - \mathbf{y}_j\|^2\right)}{\sum_{k \neq i} \exp\left(-\|\mathbf{y}_i - \mathbf{y}_k\|^2\right)}
\end{equation}

\vspace{0.15cm}
\textbf{Critical Difference: Fixed variance!}
\begin{itemize}
    \setlength\itemsep{0em}
    \item No $\sigma$ parameter in low dimensions (effectively $\sigma = 1/\sqrt{2}$)
    \item Why? We want the \emph{map} to handle density differences
    \item Points should spread out or compress naturally
    \item The optimization will arrange points to match neighborhoods
\end{itemize}

\vspace{0.15cm}
\begin{columns}
\begin{column}{0.5\textwidth}
\centering
\textbf{High-D: Adaptive $\sigma_i$}
\begin{tikzpicture}[scale=0.65]
    \fill[red] (0,0) circle (2pt);
    \node[below] at (0,-0.2) {\tiny $\mathbf{x}_i$};
    
    % Variable width kernels
    \draw[dashed, blue!50] (0,0) circle (0.7cm);
    \node[blue] at (0,-1.1) {\tiny Variable $\sigma_i$};
    
    \fill[red] (2.3,0) circle (2pt);
    \node[below] at (2.3,-0.2) {\tiny $\mathbf{x}_j$};
    \draw[dashed, blue!50] (2.3,0) circle (1.1cm);
    \node[blue] at (2.3,-1.5) {\tiny Variable $\sigma_j$};
\end{tikzpicture}
\end{column}
\begin{column}{0.5\textwidth}
\centering
\textbf{Low-D: Fixed kernel}
\begin{tikzpicture}[scale=0.65]
    \fill[red] (0,0) circle (2pt);
    \node[below] at (0,-0.2) {\tiny $\mathbf{y}_i$};
    
    % Fixed width kernels
    \draw[dashed, green!50] (0,0) circle (0.7cm);
    \node[green!70!black] at (0,-1.1) {\tiny Fixed width};
    
    \fill[red] (2.3,0) circle (2pt);
    \node[below] at (2.3,-0.2) {\tiny $\mathbf{y}_j$};
    \draw[dashed, green!50] (2.3,0) circle (0.7cm);
    \node[green!70!black] at (2.3,-1.1) {\tiny Fixed width};
\end{tikzpicture}
\end{column}
\end{columns}

\vspace{0.1cm}
\begin{block}{The Goal}
Find map points $\{\mathbf{y}_1, \ldots, \mathbf{y}_n\}$ such that $Q$ matches $P$ as closely as possible!
\end{block}

\end{frame}

%==============================================================================
% SLIDE 11: KL Divergence - Measuring the Mismatch
%==============================================================================
\begin{frame}{Measuring the Mismatch: Kullback-Leibler Divergence}

\textbf{The Core Question:} How do we measure if distribution $Q$ matches distribution $P$?

\vspace{0.2cm}
\begin{block}{The Cost Function: KL Divergence}
\begin{equation}
C = \sum_i KL(P_i \| Q_i) = \sum_i \sum_{j \neq i} p_{j|i} \log \frac{p_{j|i}}{q_{j|i}}
\end{equation}
\end{block}

\vspace{0.15cm}
\textbf{Why KL Divergence?}
\begin{itemize}
    \setlength\itemsep{0em}
    \item Measures "information lost" when using $Q$ to approximate $P$
    \item Always non-negative, zero only when $P = Q$
    \item \textbf{Crucially asymmetric}: $KL(P\|Q) \neq KL(Q\|P)$
\end{itemize}

\vspace{0.15cm}
\textbf{The Asymmetry Table—The Secret to Local Structure Preservation:}
\begin{center}
\begin{tabular}{|l|c|l|}
\hline
\textbf{Scenario} & \textbf{$p_{j|i}$ (High-D)} & \textbf{$q_{j|i}$ (Low-D)} \\
\hline
True neighbor & Large & \textcolor{green!70!black}{Must be large} \\
& & \textcolor{green!70!black}{(big penalty if small)} \\
\hline
Not neighbor & Small & \textcolor{orange}{Can be anything} \\
& & \textcolor{orange}{(small penalty regardless)} \\
\hline
\end{tabular}
\end{center}

\vspace{0.1cm}
\begin{alertblock}{Key Insight}
KL divergence heavily penalizes placing true neighbors far apart,\\
but barely penalizes placing non-neighbors close together!
\end{alertblock}

\end{frame}

%==============================================================================
% SLIDE 12: Asymmetry Visualization
%==============================================================================
\begin{frame}{The Asymmetry in Action: Why Local Structure Wins}

\begin{columns}
\begin{column}{0.5\textwidth}
\centering
\textbf{Scenario A: Missing a Neighbor}\\
\vspace{0.2cm}
\begin{tikzpicture}[scale=0.8]
    % High-D
    \node[above] at (0,1.5) {\small High-D};
    \fill[red] (0,0) circle (3pt);
    \fill[blue] (0.8,0) circle (3pt);
    \draw[<->, thick, green!70!black] (0.1,0) -- (0.7,0);
    \node[below] at (0.4,-0.3) {\tiny Close};
    
    % Arrow
    \draw[->, thick, gray] (1.5,0) -- (2.5,0);
    
    % Low-D
    \node[above] at (4,1.5) {\small Low-D};
    \fill[red] (3,0) circle (3pt);
    \fill[blue] (5,0) circle (3pt);
    \draw[<->, thick, red] (3.1,0) -- (4.9,0);
    \node[below] at (4,-0.3) {\tiny Far!};
    
    % Penalty
    \node[red, thick] at (2.5,-1) {$\boldsymbol{\times}$ Huge penalty!};
    \node at (2.5,-1.5) {\tiny $p_{j|i} \log \frac{p_{j|i}}{q_{j|i}}$ is large};
\end{tikzpicture}
\end{column}
\begin{column}{0.5\textwidth}
\centering
\textbf{Scenario B: False Neighbor}\\
\vspace{0.2cm}
\begin{tikzpicture}[scale=0.8]
    % High-D
    \node[above] at (0,1.5) {\small High-D};
    \fill[red] (0,0) circle (3pt);
    \fill[blue] (2,0) circle (3pt);
    \draw[<->, thick, red] (0.1,0) -- (1.9,0);
    \node[below] at (1,-0.3) {\tiny Far};
    
    % Arrow
    \draw[->, thick, gray] (2.5,0) -- (3.5,0);
    
    % Low-D
    \node[above] at (4.5,1.5) {\small Low-D};
    \fill[red] (4,0) circle (3pt);
    \fill[blue] (4.8,0) circle (3pt);
    \draw[<->, thick, green!70!black] (4.1,0) -- (4.7,0);
    \node[below] at (4.4,-0.3) {\tiny Close!};
    
    % Penalty
    \node[orange] at (3,-1) {$\checkmark$ Small penalty};
    \node at (3,-1.5) {\tiny $p_{j|i} \approx 0$, so penalty $\approx 0$};
\end{tikzpicture}
\end{column}
\end{columns}

\vspace{0.3cm}
\textbf{The Consequence:}
\begin{itemize}
    \setlength\itemsep{0em}
    \item Algorithm prioritizes preserving true neighborhoods
    \item Distant points can move closer if needed (crowding problem!)
    \item This is why SNE focuses on local structure
\end{itemize}

\end{frame}

%==============================================================================
% SLIDE 13: Symmetric SNE
%==============================================================================
\begin{frame}{Improving the Framework: Symmetric SNE}

\textbf{The Original SNE Problem:}
\begin{itemize}
    \setlength\itemsep{0em}
    \item Conditional probabilities: $p_{j|i} \neq p_{i|j}$ and $q_{j|i} \neq q_{i|j}$
    \item Complex gradients with $2n$ different distributions
    \item Outliers get crushed (if point $i$ is isolated, all $p_{j|i}$ are tiny)
\end{itemize}

\vspace{0.15cm}
\begin{block}{The Solution: Joint Probabilities}
Instead of conditional, use joint probabilities:
\begin{equation}
p_{ij} = \frac{p_{j|i} + p_{i|j}}{2n}, \quad \text{ensuring } \sum_{i,j} p_{ij} = 1
\end{equation}
\end{block}

\vspace{0.1cm}
\textbf{Benefits of Symmetry:}
\begin{itemize}
    \setlength\itemsep{0em}
    \item Single cost function: $C = KL(P\|Q) = \sum_{i,j} p_{ij} \log \frac{p_{ij}}{q_{ij}}$
    \item Simpler gradients: $\frac{\partial C}{\partial \mathbf{y}_i} = 4 \sum_j (p_{ij} - q_{ij})(\mathbf{y}_i - \mathbf{y}_j)$
    \item Better outlier handling: if $p_{j|i}$ is tiny but $p_{i|j}$ isn't, the average helps
    \item Each pair $(i,j)$ contributes once, not twice
\end{itemize}

\vspace{0.1cm}
\begin{exampleblock}{Practical Impact}
\footnotesize
Symmetric SNE converges faster and produces cleaner embeddings,\\
setting the stage for the final innovation: the Student-t distribution!
\end{exampleblock}

\end{frame}

%==============================================================================
% SLIDE 14: The Crowding Problem
%==============================================================================
\begin{frame}{The Fundamental Flaw: The Crowding Problem}

\begin{center}
\textbf{\large Why SNE Struggles with Well-Separated Clusters}
\end{center}

\vspace{0.1cm}
\textbf{The Volume Mismatch:}
\begin{itemize}
    \setlength\itemsep{-0.1em}
    \item In high-D: vast space to arrange moderate distances
    \item In 2D: limited space causes "traffic jam" of moderate distances
    \item Result: clusters overlap that shouldn't!
\end{itemize}

\vspace{0.1cm}
\begin{figure}
\centering
\begin{tikzpicture}[scale=0.7]
    % High-D representation
    \node[above] at (2,2.1) {\footnotesize High-D Space};
    \draw[thick] (2,0) circle (1.6cm);
    
    % Three well-separated clusters
    \fill[blue!30] (1.1,0.3) circle (0.22cm);
    \fill[red!30] (2.9,0.3) circle (0.22cm);
    \fill[green!30] (2,-0.6) circle (0.22cm);
    
    % Distance annotations
    \draw[<->, gray] (1.3,0.3) -- (2.7,0.3);
    \node[above, gray] at (2,0.3) {\tiny moderate};
    
    % Arrow to Low-D
    \draw[->, thick, red!70] (4,0) -- (5.5,0);
    \node[above] at (4.75,0) {\footnotesize SNE};
    
    % Low-D with crowding
    \node[above] at (7.5,2.1) {\footnotesize 2D Space (SNE)};
    \draw[thick] (7.5,0) circle (1.2cm);
    
    % Overlapping clusters
    \fill[blue!30, opacity=0.7] (7.1,0.15) circle (0.25cm);
    \fill[red!30, opacity=0.7] (7.8,0.15) circle (0.25cm);
    \fill[green!30, opacity=0.7] (7.45,-0.25) circle (0.25cm);
    
    \node[red] at (7.5,-1.6) {\footnotesize Clusters forced together!};
\end{tikzpicture}
\end{figure}

\vspace{0.05cm}
\textbf{The Mathematical Root:}
\vspace{-0.1cm}
\begin{center}
\footnotesize
\renewcommand{\arraystretch}{0.9}
\begin{tabular}{lcc}
\hline
\textbf{Distance Type} & \textbf{Gaussian} & \textbf{Problem} \\
\hline
Small (neighbors) & $e^{-d^2} \approx 1$ & Preserved $\checkmark$ \\
Moderate (between) & $e^{-d^2} \approx 0.1$ & \textcolor{red}{Too much weight!} \\
Large (far) & $e^{-d^2} \approx 0$ & Already zero \\
\hline
\end{tabular}
\end{center}

\end{frame}

%==============================================================================
% SLIDE 15: The t-SNE Innovation
%==============================================================================
\begin{frame}{The Breakthrough: Student-t Distribution}

\textbf{The Elegant Solution:} Replace Gaussian with Student-t (df=1) in low dimensions!

\vspace{0.15cm}
\begin{block}{The t-SNE Formula}
In low dimensions, use:
\begin{equation}
q_{ij} = \frac{(1 + ||\mathbf{y}_i - \mathbf{y}_j||^2)^{-1}}{\sum_{k \neq l} (1 + ||\mathbf{y}_k - \mathbf{y}_l||^2)^{-1}}
\end{equation}
This is the Student-t distribution with one degree of freedom (Cauchy distribution).
\end{block}

\vspace{0.1cm}
\begin{columns}
\begin{column}{0.48\textwidth}
\centering
\begin{tikzpicture}[scale=0.65]
    \begin{axis}[
        width=5.5cm, height=3.5cm,
        xlabel={\tiny Distance},
        ylabel={\tiny Similarity},
        xmin=0, xmax=5,
        ymin=0, ymax=1,
        legend pos=north east,
        legend style={font=\tiny},
        grid=major,
        grid style={gray!20},
        label style={font=\tiny},
        tick label style={font=\tiny}
    ]
    \addplot[blue, thick, domain=0:5, samples=100] {exp(-x^2)};
    \addlegendentry{Gaussian}
    
    \addplot[red, thick, domain=0:5, samples=100] {1/(1+x^2)};
    \addlegendentry{Student-t}
    
    % Highlight difference
    \draw[<->, green!70!black, thick] (axis cs:2,0.018) -- (axis cs:2,0.2);
    \node[green!70!black, right] at (axis cs:2.2,0.1) {\tiny Gap!};
    \end{axis}
\end{tikzpicture}

\vspace{0.1cm}
\textbf{Why This Works:}
\begin{itemize}
    \setlength\itemsep{0em}
    \item \textbf{Heavy tails:} slower decay
    \item \textbf{More "room":} for separation
    \item \textbf{Result:} better clusters!
\end{itemize}
\end{column}
\begin{column}{0.48\textwidth}
\centering
\textbf{Visual Impact:}\\
\vspace{0.2cm}
\begin{tikzpicture}[scale=0.55]
    % SNE result
    \node[above] at (0,2.3) {\tiny SNE (Gaussian)};
    \draw[thick] (0,0) circle (2cm);
    
    % Overlapping clusters in SNE
    \fill[blue!40, opacity=0.8] (-0.5,0.3) circle (0.5cm);
    \fill[red!40, opacity=0.8] (0.5,0.3) circle (0.5cm);
    \fill[green!40, opacity=0.8] (0,-0.5) circle (0.5cm);
    
    \node at (0,-2.5) {\tiny Crowded};
    
    % Arrow
    \draw[->, thick] (2.5,0) -- (3.5,0);
    
    % t-SNE result
    \node[above] at (6,2.3) {\tiny t-SNE (Student-t)};
    \draw[thick] (6,0) circle (2cm);
    
    % Well-separated clusters in t-SNE
    \fill[blue!70] (5.2,0.8) circle (0.35cm);
    \fill[red!70] (6.8,0.8) circle (0.35cm);
    \fill[green!70] (6,-0.9) circle (0.35cm);
    
    \node at (6,-2.5) {\tiny Well-separated};
\end{tikzpicture}
\end{column}
\end{columns}

\vspace{0.05cm}
\begin{alertblock}{The t-SNE Magic}
Heavy tails create "repulsion" between clusters while preserving local neighborhoods!
\end{alertblock}

\end{frame}

%==============================================================================
% SLIDE 16: Gradient Descent Optimization
%==============================================================================
\begin{frame}{Gradient Descent: Making t-SNE Work}

\textbf{The Optimization Problem:}
\begin{equation}
\min_{\mathbf{Y}} C = KL(P||Q) = \sum_{i,j} p_{ij} \log \frac{p_{ij}}{q_{ij}}
\end{equation}

\vspace{0.1cm}
\textbf{The Gradient (simplified t-SNE form):}
\begin{equation}
\frac{\partial C}{\partial \mathbf{y}_i} = 4 \sum_j (p_{ij} - q_{ij}) \cdot \frac{(\mathbf{y}_i - \mathbf{y}_j)}{1 + ||\mathbf{y}_i - \mathbf{y}_j||^2}
\end{equation}

\vspace{0.1cm}
\textbf{Physical Interpretation—Forces Between Points:}
\begin{itemize}
    \setlength\itemsep{-0.1em}
    \item \textbf{Attractive force:} when $p_{ij} > q_{ij}$ (should be neighbors but aren't)
    \item \textbf{Repulsive force:} when $p_{ij} < q_{ij}$ (shouldn't be neighbors but are)
    \item Points move until forces balance
\end{itemize}

\vspace{0.05cm}
\begin{figure}
\centering
\begin{tikzpicture}[scale=0.7]
    % Attractive forces
    \fill[red] (0,0) circle (3pt);
    \fill[blue] (2,0) circle (3pt);
    \draw[->, thick, green!70!black] (0.2,0) -- (1.0,0);
    \draw[->, thick, green!70!black] (1.8,0) -- (1.0,0);
    \node[above] at (1,0.2) {\tiny Attract};
    \node[below] at (1,-0.25) {\tiny $p_{ij} > q_{ij}$};
    
    % Repulsive forces
    \fill[red] (4,0) circle (3pt);
    \fill[blue] (5,0) circle (3pt);
    \draw[->, thick, red] (4.2,0) -- (3.5,0);
    \draw[->, thick, red] (4.8,0) -- (5.5,0);
    \node[above] at (4.5,0.2) {\tiny Repel};
    \node[below] at (4.5,-0.25) {\tiny $p_{ij} < q_{ij}$};
\end{tikzpicture}
\end{figure}

\vspace{0.05cm}
\begin{exampleblock}{Optimization Details}
\footnotesize
\textbf{Learning rate:} Typically 200-500 \quad \textbf{Momentum:} $\alpha$ = 0.5 → 0.8\\
\textbf{Early exaggeration:} Multiply $p_{ij}$ by 4 for first 50 iterations
\end{exampleblock}

\end{frame}

%==============================================================================
% SLIDE 17: Computational Efficiency
%==============================================================================
\begin{frame}{Computational Efficiency: Making t-SNE Scalable}

\textbf{The Computational Challenge:}
\begin{itemize}
    \setlength\itemsep{0em}
    \item Naive implementation: $O(n^2)$ for each iteration
    \item Need to compute all pairwise distances and probabilities
    \item For $n=10,000$ points: 100 million computations per iteration!
\end{itemize}

\vspace{0.2cm}
\textbf{Key Optimization: Barnes-Hut Approximation}
\begin{itemize}
    \setlength\itemsep{0em}
    \item Group distant points using quad-tree/oct-tree
    \item Treat clusters as single points for repulsive forces
    \item Reduces complexity to $O(n \log n)$
\end{itemize}

\vspace{0.15cm}
\begin{figure}
\centering
\begin{tikzpicture}[scale=0.65]
    % Exact computation
    \node[above] at (1.5,2) {\footnotesize Exact};
    \fill[red] (0,0) circle (3pt);
    \foreach \angle in {0,40,80,120,160,200,240,280,320} {
        \fill[blue] (\angle:1.5) circle (2pt);
        \draw[->, gray, opacity=0.5] (0,0) -- (\angle:1.3);
    }
    \node[below] at (0,-2) {\tiny All interactions};
    
    % Barnes-Hut
    \node[above] at (6,2) {\footnotesize Barnes-Hut};
    \fill[red] (5,0) circle (3pt);
    
    % Individual nearby points
    \foreach \angle in {0,40,80} {
        \fill[blue] (5,0) ++(\angle:0.7) circle (2pt);
        \draw[->, gray] (5,0) -- +(\angle:0.5);
    }
    
    % Grouped distant points
    \draw[dashed, orange] (6.5,0.5) circle (0.3cm);
    \fill[orange!50] (6.5,0.5) circle (2pt);
    \draw[->, orange, thick] (5.2,0.1) -- (5.8,0.3);
    \node[below] at (5,-2) {\tiny Group distant points};
\end{tikzpicture}
\end{figure}

\vspace{0.1cm}
\textbf{Speed-up:} $\frac{n}{\log n} \approx 750\times$ faster for $n=10,000$!

\end{frame}

%==============================================================================
% SLIDE 18: Hyperparameter Tuning
%==============================================================================
\begin{frame}{Hyperparameter Tuning: Getting the Best Results}

\begin{center}
\textbf{\large Critical Parameters and Their Impact}
\end{center}

\vspace{0.2cm}
\begin{columns}
\begin{column}{0.5\textwidth}
\textbf{Perplexity (5-50)}
\begin{itemize}
    \setlength\itemsep{0em}
    \item \textbf{Low (5-10):} Focus on very local structure
    \item \textbf{Medium (20-30):} Balanced view
    \item \textbf{High (40-50):} More global structure
\end{itemize}

\vspace{0.15cm}
\begin{tikzpicture}[scale=0.5]
    % Low perplexity
    \node[above] at (0,2) {\tiny Perp=5};
    \foreach \i in {1,...,3} {
        \foreach \j in {1,...,3} {
            \fill[blue!60] (\i*0.3,\j*0.3) circle (2pt);
        }
    }
    \foreach \i in {1,...,3} {
        \foreach \j in {1,...,3} {
            \fill[red!60] (\i*0.3+1.5,\j*0.3) circle (2pt);
        }
    }
    \node[below] at (1.5,-0.5) {\tiny Many clusters};
    
    % High perplexity
    \node[above] at (5,2) {\tiny Perp=50};
    \fill[blue!60] (4.5,0.8) circle (0.4cm);
    \fill[red!60] (5.5,0.8) circle (0.4cm);
    \node[below] at (5,-0.5) {\tiny Merged clusters};
\end{tikzpicture}
\end{column}
\begin{column}{0.5\textwidth}
\textbf{Learning Rate (100-1000)}
\begin{itemize}
    \setlength\itemsep{0em}
    \item \textbf{Too low:} Stuck in bad minima
    \item \textbf{Too high:} Points explode
    \item \textbf{Auto-scaling:} $\eta = n/12$ works well
\end{itemize}

\vspace{0.15cm}
\textbf{Number of Iterations}
\begin{itemize}
    \setlength\itemsep{0em}
    \item Minimum: 250 iterations
    \item Standard: 1000 iterations
    \item Large/complex data: 5000+
\end{itemize}
\end{column}
\end{columns}

\vspace{0.2cm}
\begin{alertblock}{Rule of Thumb for Perplexity}
\footnotesize
Perplexity should be between 5 and 50, typically $\sqrt{n}/20$ to $\sqrt{n}/5$ where $n$ is sample size.\\
Try multiple values: if consistent patterns appear across different perplexities, they're likely real.
\end{alertblock}

\vspace{0.15cm}
\begin{exampleblock}{Early Exaggeration Trick}
\footnotesize
Multiply all $p_{ij}$ by 4-12 in early iterations to form tight clusters early.\\
This helps separate clusters that should be distinct from the start.
\end{exampleblock}

\end{frame}

%==============================================================================
% SLIDE 19: Interpreting t-SNE Results
%==============================================================================
\begin{frame}{Critical: How to (Not) Interpret t-SNE}

\begin{center}
\textbf{\Large What You CAN and CANNOT Trust}
\end{center}

\vspace{0.2cm}
\begin{columns}
\begin{column}{0.48\textwidth}
\textbf{\textcolor{green!70!black}{CAN Trust:}}
\begin{itemize}
    \setlength\itemsep{0em}
    \item[$\checkmark$] Local neighborhoods
    \item[$\checkmark$] Which points cluster together
    \item[$\checkmark$] Presence of clusters
    \item[$\checkmark$] Relative density within clusters
\end{itemize}

\vspace{0.2cm}
\begin{tikzpicture}[scale=0.7]
    \draw[green!70!black, thick, dashed] (-0.5,-0.5) rectangle (3.5,1.5);
    \fill[blue!60] (0,0) circle (0.3cm);
    \fill[blue!60] (1,0.5) circle (0.3cm);
    \fill[blue!60] (0.8,0) circle (0.3cm);
    \node[green!70!black] at (1.5,-0.8) {\tiny These cluster together};
\end{tikzpicture}
\end{column}
\begin{column}{0.48\textwidth}
\textbf{\textcolor{red}{CANNOT Trust:}}
\begin{itemize}
    \setlength\itemsep{0em}
    \item[$\boldsymbol{\times}$] Distances between clusters
    \item[$\boldsymbol{\times}$] Cluster sizes (area)
    \item[$\boldsymbol{\times}$] Cluster shapes
    \item[$\boldsymbol{\times}$] Exact positions
\end{itemize}

\vspace{0.2cm}
\begin{tikzpicture}[scale=0.7]
    \fill[blue!60] (0,0) circle (0.2cm);
    \fill[red!60] (2.5,0) circle (0.4cm);
    \draw[<->, red, thick] (0.3,0) -- (2.1,0);
    \node[red] at (1.2,-0.5) {\tiny Meaningless!};
    \draw[red, thick] (2.9,0.3) -- (3.2,0.6);
    \draw[red, thick] (2.9,-0.3) -- (3.2,-0.6);
    \node[red] at (3.5,0) {\tiny Size?};
\end{tikzpicture}
\end{column}
\end{columns}

\vspace{0.2cm}
\textbf{Common Misinterpretations to Avoid:}
\begin{itemize}
    \setlength\itemsep{0em}
    \item "Cluster A is twice as big as B" — \textcolor{red}{NO!} Sizes are arbitrary
    \item "These clusters are far apart" — \textcolor{red}{NO!} Inter-cluster distance meaningless
    \item "This cluster has a moon shape" — \textcolor{red}{NO!} Shapes are distorted
    \item "The center point is special" — \textcolor{red}{NO!} Positions are relative
\end{itemize}

\vspace{0.15cm}
\begin{alertblock}{Golden Rule}
t-SNE preserves \textbf{who is your neighbor}, not \textbf{how far away they are}!
\end{alertblock}

\end{frame}

%==============================================================================
% SLIDE 20: R Implementation Overview
%==============================================================================
\begin{frame}{R Implementation: Getting Started}

\begin{center}
\textbf{\Large Running t-SNE in R}
\end{center}

\vspace{0.2cm}
\textbf{Required Package:}
\begin{itemize}
    \item \texttt{Rtsne} — Main t-SNE implementation
    \item Install: \texttt{install.packages("Rtsne")}
\end{itemize}

\vspace{0.2cm}
\textbf{Basic Workflow:}
\begin{enumerate}
    \setlength\itemsep{0em}
    \item \textbf{Prepare data:} Numeric matrix, remove duplicates
    \item \textbf{Choose parameters:} Perplexity, iterations, learning rate
    \item \textbf{Run t-SNE:} \texttt{Rtsne()} function
    \item \textbf{Visualize:} Extract Y coordinates and plot
    \item \textbf{Validate:} Try different perplexities
\end{enumerate}

\vspace{0.2cm}
\begin{exampleblock}{Scripts for Today's Session}
\footnotesize
\textbf{Script 1:} \texttt{01\_tsne\_basics.R} — Basic t-SNE on iris dataset\\
\textbf{Script 2:} \texttt{02\_parameter\_exploration.R} — Effect of perplexity\\
\textbf{Script 3:} \texttt{03\_mnist\_example.R} — Larger dataset (MNIST subset)\\
\textbf{Script 4:} \texttt{04\_comparison\_methods.R} — t-SNE vs PCA vs MDS
\end{exampleblock}

\vspace{0.15cm}
\textbf{Key Parameters to Remember:}
\begin{itemize}
    \setlength\itemsep{0em}
    \item \texttt{perplexity}: 5-50 (default 30)
    \item \texttt{max\_iter}: 1000+ for convergence
    \item \texttt{set.seed()}: Always set for reproducibility
\end{itemize}

\end{frame}


%==============================================================================
% SLIDE 21: Gradient Components Deep Dive
%==============================================================================
\begin{frame}{Deep Dive: The Gradient Components}

\textbf{Breaking Down the t-SNE Gradient:}

\vspace{0.2cm}
The gradient has two competing components:

\begin{equation}
\frac{\partial C}{\partial \mathbf{y}_i} = 4 \sum_j \underbrace{(p_{ij} - q_{ij})}_{\text{error}} \cdot \underbrace{\frac{(\mathbf{y}_i - \mathbf{y}_j)}{1 + ||\mathbf{y}_i - \mathbf{y}_j||^2}}_{\text{direction \& magnitude}}
\end{equation}

\vspace{0.2cm}
\textbf{Component Analysis:}
\begin{itemize}
    \setlength\itemsep{0em}
    \item \textbf{Error term} $(p_{ij} - q_{ij})$: How wrong is our current embedding?
    \item \textbf{Direction} $(\mathbf{y}_i - \mathbf{y}_j)$: Which way to move point $i$?
    \item \textbf{Magnitude modifier} $\frac{1}{1 + ||\mathbf{y}_i - \mathbf{y}_j||^2}$: Student-t kernel
\end{itemize}

\vspace{0.2cm}
\textbf{Why the factor of 4?}
\begin{itemize}
    \setlength\itemsep{0em}
    \item Comes from symmetrization: $p_{ij} = \frac{p_{j|i} + p_{i|j}}{2n}$
    \item Derivative of Student-t: $\frac{\partial}{\partial y_i}(1 + d^2)^{-1} = -2y_i(1 + d^2)^{-2}$
    \item Combined with normalization constants → factor of 4
\end{itemize}

\end{frame}

%==============================================================================
% SLIDE 22: Early Exaggeration Details
%==============================================================================
\begin{frame}{Early Exaggeration: Forcing Initial Separation}

\textbf{The Technique:}
\begin{equation}
p_{ij}^{\text{exaggerated}} = \alpha \cdot p_{ij}, \quad \text{where } \alpha = 4 \text{ or } 12
\end{equation}

\vspace{0.2cm}
\textbf{Timeline:}
\begin{itemize}
    \setlength\itemsep{0em}
    \item Iterations 1-50: Apply exaggeration ($\alpha = 12$ recommended)
    \item Iterations 51-250: Remove exaggeration, let settle
    \item Iterations 251+: Fine-tune positions
\end{itemize}

\vspace{0.2cm}
\begin{figure}
\centering
\begin{tikzpicture}[scale=0.8]
    % Before exaggeration
    \node[above] at (1.5,1.5) {\small Without Exaggeration};
    \draw[thick] (1.5,0) circle (1.2cm);
    \fill[blue!40, opacity=0.7] (1.2,0.2) circle (0.3cm);
    \fill[red!40, opacity=0.7] (1.8,0.1) circle (0.3cm);
    \fill[green!40, opacity=0.7] (1.5,-0.2) circle (0.3cm);
    
    % Arrow
    \draw[->, thick] (3,0) -- (4,0);
    \node[above] at (3.5,0) {\small $\alpha = 12$};
    
    % After exaggeration
    \node[above] at (6,1.5) {\small With Exaggeration};
    \draw[thick] (6,0) circle (1.2cm);
    \fill[blue!70] (5.3,0.5) circle (0.2cm);
    \fill[red!70] (6.7,0.4) circle (0.2cm);
    \fill[green!70] (6,-0.6) circle (0.2cm);
\end{tikzpicture}
\end{figure}

\vspace{0.1cm}
\textbf{Why It Works:}
Forces tight clusters early, preventing them from mixing during optimization

\end{frame}

%==============================================================================
% SLIDE 23: Momentum and Learning Rate
%==============================================================================
\begin{frame}{Advanced Optimization: Momentum \& Adaptive Learning}

\textbf{Momentum Update Rule:}
\begin{equation}
\mathbf{y}_i^{(t+1)} = \mathbf{y}_i^{(t)} + \eta \frac{\partial C}{\partial \mathbf{y}_i} + \alpha(t)(\mathbf{y}_i^{(t)} - \mathbf{y}_i^{(t-1)})
\end{equation}

\vspace{0.15cm}
\textbf{Momentum Schedule (van der Maaten 2014):}
\begin{itemize}
    \setlength\itemsep{0em}
    \item $t < 250$: $\alpha = 0.5$ (low momentum during early exaggeration)
    \item $t \geq 250$: $\alpha = 0.8$ (high momentum for fine-tuning)
\end{itemize}

\vspace{0.15cm}
\textbf{Adaptive Learning Rate:}
\begin{itemize}
    \setlength\itemsep{0em}
    \item If cost improves: $\eta^{(t+1)} = \eta^{(t)} \times 1.01$
    \item If cost worsens: $\eta^{(t+1)} = \eta^{(t)} \times 0.5$
    \item Minimum: $\eta_{\min} = 10$
\end{itemize}

\vspace{0.15cm}
\begin{exampleblock}{Implementation Tip}
\footnotesize
Most modern implementations use adaptive learning with gradient clipping:\\
\texttt{grad\_norm = max(1, ||gradient||/threshold)}\\
\texttt{gradient = gradient / grad\_norm}
\end{exampleblock}

\end{frame}

%==============================================================================
% SLIDE 24: Barnes-Hut Algorithm Details
%==============================================================================
\begin{frame}{Barnes-Hut Approximation: The Technical Details}

\textbf{The Core Idea:} Use a tree structure to group distant points

\vspace{0.15cm}
\textbf{Quadtree Construction (2D):}
\begin{enumerate}
    \setlength\itemsep{0em}
    \item Recursively divide space into quadrants
    \item Stop when cell contains $\leq 1$ point
    \item Store center of mass for each cell
\end{enumerate}

\vspace{0.15cm}
\begin{figure}
\centering
\begin{tikzpicture}[scale=0.7]
    % Main square
    \draw[thick] (0,0) rectangle (4,4);
    
    % Quadrant divisions
    \draw (2,0) -- (2,4);
    \draw (0,2) -- (4,2);
    
    % Further subdivision in one quadrant
    \draw (3,0) -- (3,2);
    \draw (2,1) -- (4,1);
    
    % Points
    \fill[red] (0.5,3.5) circle (2pt);
    \fill[blue] (1.5,2.5) circle (2pt);
    \fill[blue] (1.2,2.8) circle (2pt);
    \fill[green] (3.5,1.5) circle (2pt);
    \fill[green] (3.2,1.3) circle (2pt);
    \fill[green] (3.7,1.7) circle (2pt);
    \fill[orange] (2.5,0.5) circle (2pt);
    
    % Center of mass for grouped points
    \fill[green!50] (3.47,1.5) circle (3pt);
    \draw[green, thick] (3.47,1.5) circle (0.4cm);
    
    % Annotation
    \node[right] at (4.5,2) {\small Cell with};
    \node[right] at (4.5,1.5) {\small multiple points};
    \draw[->] (4.5,1.5) -- (3.9,1.5);
\end{tikzpicture}
\end{figure}

\vspace{0.15cm}
\textbf{Barnes-Hut Criterion:}
\begin{equation}
\text{If } \frac{\text{cell\_width}}{||\mathbf{y}_i - \mathbf{cell\_center}||} < \theta \text{ then treat as single point}
\end{equation}

Typical $\theta = 0.5$ (trade-off: speed vs accuracy)

\end{frame}

%==============================================================================
% SLIDE 25: Initialization Strategies
%==============================================================================
\begin{frame}{Initialization Strategies: Starting Smart}

\textbf{Three Main Approaches:}

\vspace{0.2cm}
\textbf{1. Random Initialization}
\begin{itemize}
    \setlength\itemsep{0em}
    \item Sample from $\mathcal{N}(0, 10^{-4})$ 
    \item Simple but unstable
    \item Different runs → different results
\end{itemize}

\vspace{0.15cm}
\textbf{2. PCA Initialization (Recommended)}
\begin{itemize}
    \setlength\itemsep{0em}
    \item Start with first 2 principal components
    \item Preserves global structure initially
    \item Faster convergence (30-50\% fewer iterations)
    \item More reproducible results
\end{itemize}

\vspace{0.15cm}
\textbf{3. Multiple Random Restarts}
\begin{itemize}
    \setlength\itemsep{0em}
    \item Run t-SNE 5-10 times
    \item Choose embedding with lowest KL divergence
    \item Computationally expensive but robust
\end{itemize}

\vspace{0.15cm}
\begin{figure}
\centering
\begin{tikzpicture}[scale=0.6]
    % Random init
    \node[above] at (0,2) {\tiny Random};
    \foreach \i in {1,...,20} {
        \pgfmathsetmacro{\x}{rand*2}
        \pgfmathsetmacro{\y}{rand*2}
        \fill[blue!50] (\x,\y) circle (1pt);
    }
    
    % PCA init
    \node[above] at (4,2) {\tiny PCA};
    \foreach \i in {0,0.2,...,1.8} {
        \fill[blue!50] (\i+3.5,\i*0.3) circle (1pt);
        \fill[red!50] (\i+3.5,1-\i*0.2) circle (1pt);
    }
    
    % Multiple runs
    \node[above] at (8,2) {\tiny Multiple};
    \foreach \j in {0,0.5,1} {
        \foreach \i in {1,...,5} {
            \pgfmathsetmacro{\x}{rand*0.5+7.5+\j}
            \pgfmathsetmacro{\y}{rand*0.5+\j*0.5}
            \fill[blue!30] (\x,\y) circle (1pt);
        }
    }
\end{tikzpicture}
\end{figure}

\end{frame}

%==============================================================================
% SLIDE 26: Quality Metrics for t-SNE
%==============================================================================
\begin{frame}{Assessing t-SNE Quality: Beyond Visual Inspection}

\textbf{Quantitative Metrics:}

\vspace{0.2cm}
\textbf{1. KL Divergence (Built-in)}
\begin{itemize}
    \setlength\itemsep{0em}
    \item Final cost value from optimization
    \item Lower is better, but scale depends on data
    \item Useful for comparing different runs
\end{itemize}

\vspace{0.15cm}
\textbf{2. Trustworthiness \& Continuity}
\begin{equation}
T(k) = 1 - \frac{2}{nk(2n-3k-1)} \sum_i \sum_{j \in U_i^{(k)}} (r_{ij} - k)
\end{equation}
where $U_i^{(k)}$ = points in k-NN in 2D but not in high-D

\vspace{0.15cm}
\textbf{3. Neighborhood Preservation}
\begin{itemize}
    \setlength\itemsep{0em}
    \item \% of k-NN preserved from high-D to low-D
    \item Typically check k = 5, 10, 20, 50
    \item Good embeddings: >80\% for k=10
\end{itemize}

\vspace{0.15cm}
\textbf{4. Silhouette Score}
\begin{itemize}
    \setlength\itemsep{0em}
    \item Measures cluster separation quality
    \item Range: [-1, 1], higher is better
    \item Only valid if true clusters exist
\end{itemize}

\end{frame}

%==============================================================================
% SLIDE 27: Domain Applications - Genomics
%==============================================================================
\begin{frame}{Application: Single-Cell Genomics}

\textbf{The Challenge:} Visualizing 20,000+ genes across 100,000+ cells

\vspace{0.2cm}
\textbf{Typical Workflow:}
\begin{enumerate}
    \setlength\itemsep{0em}
    \item \textbf{Preprocessing:} Log-normalize, select highly variable genes
    \item \textbf{Dimensionality reduction:} PCA to 50 components
    \item \textbf{t-SNE:} On PCA output, perplexity 30-100
    \item \textbf{Clustering:} Identify cell types
    \item \textbf{Annotation:} Map clusters to biological meaning
\end{enumerate}

\vspace{0.2cm}
\begin{figure}
\centering
\begin{tikzpicture}[scale=0.8]
    % Simulated single-cell clusters
    \node[above] at (3,3) {\small Cell Type Discovery};
    
    % T cells
    \fill[blue!60] (1,1.5) circle (0.4cm);
    \node[below] at (1,1) {\tiny T cells};
    
    % B cells
    \fill[red!60] (2.5,2) circle (0.35cm);
    \node[below] at (2.5,1.5) {\tiny B cells};
    
    % NK cells
    \fill[green!60] (4,1.2) circle (0.3cm);
    \node[below] at (4,0.8) {\tiny NK cells};
    
    % Monocytes
    \fill[orange!60] (3,0.3) circle (0.45cm);
    \node[below] at (3,-0.2) {\tiny Monocytes};
    
    % Dendritic
    \fill[purple!60] (5,2.2) circle (0.25cm);
    \node[right] at (5.3,2.2) {\tiny Dendritic};
\end{tikzpicture}
\end{figure}

\vspace{0.1cm}
\textbf{Key Considerations:}
\begin{itemize}
    \setlength\itemsep{0em}
    \item Use higher perplexity (50-100) for large datasets
    \item Often combined with UMAP for speed
    \item Batch effects can create false clusters
\end{itemize}

\end{frame}

%==============================================================================
% SLIDE 28: Application - Natural Language Processing
%==============================================================================
\begin{frame}{Application: Word Embeddings \& Document Analysis}

\textbf{Visualizing Word Vectors (Word2Vec, GloVe):}

\vspace{0.2cm}
\begin{columns}
\begin{column}{0.5\textwidth}
\textbf{Setup:}
\begin{itemize}
    \setlength\itemsep{0em}
    \item Input: 300-dimensional word vectors
    \item Goal: Discover semantic relationships
    \item Perplexity: 20-40
\end{itemize}

\vspace{0.2cm}
\textbf{What t-SNE Reveals:}
\begin{itemize}
    \setlength\itemsep{0em}
    \item Semantic clusters (animals, countries, verbs)
    \item Analogical relationships
    \item Polysemy (multiple meanings)
\end{itemize}
\end{column}
\begin{column}{0.5\textwidth}
\begin{tikzpicture}[scale=0.8]
    % Countries cluster
    \draw[blue!30, thick] (0.5,2) circle (0.8cm);
    \node at (0.3,2.2) {\tiny France};
    \node at (0.7,2) {\tiny Germany};
    \node at (0.5,1.8) {\tiny Spain};
    
    % Animals cluster
    \draw[green!30, thick] (3,2) circle (0.7cm);
    \node at (2.8,2.2) {\tiny cat};
    \node at (3.2,2) {\tiny dog};
    \node at (3,1.8) {\tiny mouse};
    
    % Verbs cluster
    \draw[red!30, thick] (2,0.5) circle (0.9cm);
    \node at (1.8,0.7) {\tiny run};
    \node at (2.2,0.5) {\tiny walk};
    \node at (2,0.3) {\tiny jump};
\end{tikzpicture}
\end{column}
\end{columns}

\vspace{0.2cm}
\textbf{Document Clustering:}
\begin{itemize}
    \setlength\itemsep{0em}
    \item TF-IDF or Doc2Vec representations
    \item Reveals topic clusters
    \item Useful for corpus exploration
\end{itemize}

\vspace{0.15cm}
\begin{alertblock}{NLP Tip}
Pre-reduce to 50-100 dims with PCA/SVD before t-SNE for better results
\end{alertblock}

\end{frame}

%==============================================================================
% SLIDE 29: Application - Computer Vision
%==============================================================================
\begin{frame}{Application: Deep Learning Feature Visualization}

\textbf{Visualizing CNN Representations:}

\vspace{0.2cm}
\textbf{Common Use Cases:}
\begin{itemize}
    \setlength\itemsep{0em}
    \item Final layer features (before softmax)
    \item Intermediate layer activations
    \item Learned embeddings (face recognition, image retrieval)
\end{itemize}

\vspace{0.2cm}
\begin{figure}
\centering
\begin{tikzpicture}[scale=0.9]
    % CNN Architecture
    \node[rectangle, draw, minimum width=0.5cm, minimum height=1.5cm] at (0,0) {\tiny Input};
    \node[rectangle, draw, minimum width=0.8cm, minimum height=1.2cm] at (1.5,0) {\tiny Conv};
    \node[rectangle, draw, minimum width=0.8cm, minimum height=1cm] at (3,0) {\tiny Conv};
    \node[rectangle, draw, minimum width=0.6cm, minimum height=0.8cm] at (4.5,0) {\tiny FC};
    
    % Arrows
    \draw[->] (0.3,0) -- (1.1,0);
    \draw[->] (1.9,0) -- (2.6,0);
    \draw[->] (3.4,0) -- (4.1,0);
    \draw[->] (4.8,0) -- (5.5,0);
    
    % t-SNE arrow
    \draw[->, thick, red] (5.5,0) -- (6.5,-0.5);
    \node[red] at (6,-0.2) {\tiny t-SNE};
    
    % Visualization
    \draw[thick] (7,-1.5) rectangle (9,0.5);
    \foreach \i in {1,...,5} {
        \pgfmathsetmacro{\x}{rand*1.5+7.2}
        \pgfmathsetmacro{\y}{rand*1.5-1.2}
        \fill[blue!60] (\x,\y) circle (2pt);
    }
    \foreach \i in {1,...,5} {
        \pgfmathsetmacro{\x}{rand*1.5+7.2}
        \pgfmathsetmacro{\y}{rand*1.5-1.2}
        \fill[red!60] (\x,\y) circle (2pt);
    }
\end{tikzpicture}
\end{figure}

\vspace{0.2cm}
\textbf{Example: MNIST Digit Recognition}
\begin{itemize}
    \setlength\itemsep{0em}
    \item Extract 128-dim features from trained network
    \item t-SNE with perplexity $30-50$
    \item Results: Clear digit clusters
    \item Reveals confusions (e.g., 4 $\leftrightarrow$ 9, 3 $\leftrightarrow$ 8)
\end{itemize}

\vspace{0.1cm}
\textbf{Best Practices:}
\begin{itemize}
    \setlength\itemsep{0em}
    \item Use features before final classification layer
    \item Normalize features before t-SNE
    \item Color by true labels to assess quality
\end{itemize}

\end{frame}

%==============================================================================
% SLIDE 30: Common Pitfalls and Solutions
%==============================================================================
\begin{frame}{Common Pitfalls: What Can Go Wrong}

\textbf{1. Points Collapsing to Center}
\begin{itemize}
    \setlength\itemsep{0em}
    \item \textbf{Cause:} Learning rate too high
    \item \textbf{Solution:} Reduce learning rate, use gradient clipping
\end{itemize}

\vspace{0.15cm}
\textbf{2. Poor Separation Despite Clear Clusters}
\begin{itemize}
    \setlength\itemsep{0em}
    \item \textbf{Cause:} Perplexity too high
    \item \textbf{Solution:} Try perplexity $\in [5, \sqrt{n}/5]$
\end{itemize}

\vspace{0.15cm}
\textbf{3. Different Runs Give Different Results}
\begin{itemize}
    \setlength\itemsep{0em}
    \item \textbf{Cause:} Random initialization
    \item \textbf{Solution:} Set seed, use PCA initialization, run multiple times
\end{itemize}

\vspace{0.15cm}
\textbf{4. Outliers Dominating Visualization}
\begin{itemize}
    \setlength\itemsep{0em}
    \item \textbf{Cause:} Outliers have very different scales
    \item \textbf{Solution:} Remove outliers or use robust scaling
\end{itemize}

\vspace{0.15cm}
\textbf{5. Slow Convergence}
\begin{itemize}
    \setlength\itemsep{0em}
    \item \textbf{Cause:} Too many iterations on bad initialization
    \item \textbf{Solution:} Use PCA init, early exaggeration, momentum
\end{itemize}

\end{frame}

%==============================================================================
% SLIDE 31: Complexity Analysis Detailed
%==============================================================================
\begin{frame}{Computational Complexity: Detailed Analysis}

\textbf{Naive t-SNE Complexity:}
\begin{center}
\begin{tabular}{lcc}
\hline
\textbf{Operation} & \textbf{Complexity} & \textbf{For $n=10,000$} \\
\hline
Compute all $p_{ij}$ & $O(n^2)$ & 100M operations \\
Compute all $q_{ij}$ per iteration & $O(n^2)$ & 100M operations \\
Gradient computation per iteration & $O(n^2)$ & 100M operations \\
Total for 1000 iterations & $O(n^2 \cdot T)$ & 100B operations \\
\hline
\end{tabular}
\end{center}

\vspace{0.2cm}
\textbf{Barnes-Hut t-SNE Complexity:}
\begin{center}
\begin{tabular}{lcc}
\hline
\textbf{Operation} & \textbf{Complexity} & \textbf{For $n=10,000$} \\
\hline
Build quadtree & $O(n \log n)$ & 130K operations \\
Compute forces & $O(n \log n)$ & 130K operations \\
Total for 1000 iterations & $O(n \log n \cdot T)$ & 130M operations \\
\hline
\end{tabular}
\end{center}

\vspace{0.2cm}
\textbf{Speed-up factor:} $\frac{n}{\log n} \approx 750\times$ faster for $n=10,000$!

\vspace{0.1cm}
\textbf{Memory Requirements:}
\begin{itemize}
    \setlength\itemsep{0em}
    \item Naive: $O(n^2)$ to store all pairwise distances
    \item Barnes-Hut: $O(n)$ for tree structure
\end{itemize}

\end{frame}

%==============================================================================
% SLIDE 32: Modern Scaling Approaches
%==============================================================================
\begin{frame}{Scaling to Millions: Modern Approaches}

\textbf{Challenge:} Even Barnes-Hut struggles with $n > 100,000$

\vspace{0.2cm}
\textbf{Solution 1: Landmark-based t-SNE}
\begin{itemize}
    \setlength\itemsep{0em}
    \item Select $m \ll n$ landmark points
    \item Embed landmarks with regular t-SNE
    \item Project remaining points using interpolation
    \item Complexity: $O(m^2 \log m + nm)$
\end{itemize}

\vspace{0.15cm}
\textbf{Solution 2: FFT-accelerated t-SNE (FIt-SNE)}
\begin{itemize}
    \setlength\itemsep{0em}
    \item Use Fast Fourier Transform for gradient computation
    \item Interpolation on regular grid
    \item Complexity: $O(n \log n)$ with better constants
    \item 10-100× faster than Barnes-Hut for large $n$
\end{itemize}

\vspace{0.15cm}
\textbf{Solution 3: Parametric t-SNE}
\begin{itemize}
    \setlength\itemsep{0em}
    \item Learn neural network $f: \mathbb{R}^d \rightarrow \mathbb{R}^2$
    \item Can embed new points without re-running
    \item Good for streaming data
\end{itemize}

\vspace{0.1cm}
\begin{alertblock}{Recommendation}
\footnotesize
For $n < 10,000$: Standard Barnes-Hut\\
For $10,000 < n < 1M$: FIt-SNE\\
For $n > 1M$: Landmark or parametric approaches
\end{alertblock}

\end{frame}

%==============================================================================
% SLIDE 33: Alternative Divergences
%==============================================================================
\begin{frame}{Beyond KL Divergence: Alternative Formulations}

\textbf{Original t-SNE:} Kullback-Leibler divergence
\begin{equation}
C_{KL} = \sum_{i,j} p_{ij} \log \frac{p_{ij}}{q_{ij}}
\end{equation}

\vspace{0.15cm}
\textbf{Alternative 1: Jensen-Shannon Divergence}
\begin{equation}
C_{JS} = \frac{1}{2}KL(P||M) + \frac{1}{2}KL(Q||M), \quad M = \frac{P+Q}{2}
\end{equation}
\begin{itemize}
    \item Symmetric measure
    \item More stable gradients
    \item Used in JS-SNE variant
\end{itemize}

\vspace{0.15cm}
\textbf{Alternative 2: Wasserstein Distance}
\begin{equation}
C_{W} = \inf_{\gamma \in \Gamma(P,Q)} \sum_{i,j} \gamma_{ij} \cdot d(\mathbf{y}_i, \mathbf{y}_j)
\end{equation}
\begin{itemize}
    \item Considers geometry of space
    \item Better for imbalanced clusters
    \item Computationally more expensive
\end{itemize}

\vspace{0.1cm}
\textbf{Research Note:} Most variants show marginal improvements over KL

\end{frame}

%==============================================================================
% SLIDE 34: Distribution Comparisons
%==============================================================================
\begin{frame}{Distribution Choice: Why Student-t with df=1?}

\textbf{Tested Distributions in Low-D Space:}

\vspace{0.15cm}
\begin{center}
\small
\begin{tabular}{lcc}
\hline
\textbf{Distribution} & \textbf{Formula} & \textbf{Tail Decay} \\
\hline
Gaussian & $e^{-d^2}$ & Exponential \\
Student-t (df=1) & $(1 + d^2)^{-1}$ & $O(d^{-2})$ \\
Student-t (df=2) & $(1 + d^2/2)^{-3/2}$ & $O(d^{-3})$ \\
Exponential & $e^{-d}$ & Exponential \\
\hline
\end{tabular}
\end{center}

\vspace{0.2cm}
\begin{figure}
\centering
\begin{tikzpicture}[scale=0.8]
    \begin{axis}[
        width=8cm, height=4cm,
        xlabel={\footnotesize Distance},
        ylabel={\footnotesize Similarity},
        xmin=0, xmax=5,
        ymin=0, ymax=1,
        legend pos=north east,
        legend style={font=\tiny},
        grid=major,
        grid style={gray!20}
    ]
    \addplot[blue, thick, domain=0:5, samples=100] {exp(-x^2)};
    \addlegendentry{Gaussian}
    
    \addplot[red, thick, domain=0:5, samples=100] {1/(1+x^2)};
    \addlegendentry{t (df=1)}
    
    \addplot[green, thick, domain=0:5, samples=100] {(1+x^2/2)^(-1.5)};
    \addlegendentry{t (df=2)}
    
    \addplot[orange, thick, domain=0:5, samples=100] {exp(-x)};
    \addlegendentry{Exponential}
    \end{axis}
\end{tikzpicture}
\end{figure}

\textbf{Winner:} Student-t (df=1) provides optimal balance

\end{frame}

%==============================================================================
% SLIDE 35: UMAP vs t-SNE
%==============================================================================
\begin{frame}{Modern Alternative: UMAP vs t-SNE}

\textbf{UMAP (Uniform Manifold Approximation and Projection):}

\vspace{0.15cm}
\begin{columns}
\begin{column}{0.5\textwidth}
\textbf{Similarities:}
\begin{itemize}
    \setlength\itemsep{0em}
    \item Non-linear embedding
    \item Preserves local structure
    \item Uses graph-based approach
    \item Similar visual results
\end{itemize}

\vspace{0.15cm}
\textbf{UMAP Advantages:}
\begin{itemize}
    \setlength\itemsep{0em}
    \item Faster: $O(n^{1.14})$ complexity
    \item Preserves more global structure
    \item Theoretical foundation (topology)
    \item Supports embedding new points
\end{itemize}
\end{column}
\begin{column}{0.5\textwidth}
\textbf{t-SNE Advantages:}
\begin{itemize}
    \setlength\itemsep{0em}
    \item More established/studied
    \item Better cluster separation
    \item Clearer interpretation
    \item Robust to hyperparameters
\end{itemize}

\vspace{0.15cm}
\textbf{Key Difference:}\\
\footnotesize
t-SNE: Probabilistic approach\\
UMAP: Topological approach\\

\vspace{0.15cm}
\textbf{When to Use Which:}\\
\footnotesize
t-SNE: Final visualization\\
UMAP: Exploratory analysis
\end{column}
\end{columns}

\vspace{0.15cm}
\begin{alertblock}{Current Consensus}
\footnotesize
Both are valuable tools. t-SNE for publication-quality cluster visualization,\\
UMAP for interactive exploration and when global structure matters.
\end{alertblock}

\end{frame}

%==============================================================================
% SLIDE 36: Mathematical Proof - Gradient Derivation
%==============================================================================
\begin{frame}{Mathematical Deep Dive: Gradient Derivation}

\textbf{Starting from the cost function:}
\begin{equation}
C = \sum_{i,j} p_{ij} \log \frac{p_{ij}}{q_{ij}}
\end{equation}

\textbf{Step 1:} Express $q_{ij}$ with Student-t:
\begin{equation}
q_{ij} = \frac{(1 + ||\mathbf{y}_i - \mathbf{y}_j||^2)^{-1}}{\sum_{k \neq l} (1 + ||\mathbf{y}_k - \mathbf{y}_l||^2)^{-1}} = \frac{w_{ij}}{Z}
\end{equation}

\textbf{Step 2:} Take derivative w.r.t. $\mathbf{y}_i$:
\begin{equation}
\frac{\partial C}{\partial \mathbf{y}_i} = \sum_j p_{ij} \frac{\partial \log q_{ij}}{\partial \mathbf{y}_i}
\end{equation}

\textbf{Step 3:} Apply chain rule:
\begin{equation}
\frac{\partial \log q_{ij}}{\partial \mathbf{y}_i} = \frac{1}{q_{ij}} \frac{\partial q_{ij}}{\partial \mathbf{y}_i}
\end{equation}

\textbf{Step 4:} Final gradient (after simplification):
\begin{equation}
\frac{\partial C}{\partial \mathbf{y}_i} = 4 \sum_j (p_{ij} - q_{ij}) w_{ij} (\mathbf{y}_i - \mathbf{y}_j)
\end{equation}

\end{frame}

%==============================================================================
% SLIDE 37: Convergence Properties
%==============================================================================
\begin{frame}{Convergence Properties and Guarantees}

\textbf{What We Can Prove:}

\vspace{0.2cm}
\textbf{1. Cost Function Properties}
\begin{itemize}
    \setlength\itemsep{0em}
    \item KL divergence is convex in $q_{ij}$
    \item But non-convex in $\mathbf{y}_i$ (multiple local minima)
    \item Bounded below by 0
\end{itemize}

\vspace{0.15cm}
\textbf{2. Gradient Descent Convergence}
\begin{itemize}
    \setlength\itemsep{0em}
    \item With small enough learning rate: guaranteed to converge
    \item Converges to local minimum (not necessarily global)
    \item Early exaggeration helps escape poor local minima
\end{itemize}

\vspace{0.15cm}
\textbf{3. Asymptotic Behavior}
\begin{itemize}
    \setlength\itemsep{0em}
    \item As $t \rightarrow \infty$: $||d\mathbf{y}_i/dt|| \rightarrow 0$
    \item Points reach equilibrium configuration
    \item Forces balance: $\sum_j F_{ij} = 0$ for all $i$
\end{itemize}

\vspace{0.15cm}
\textbf{What We Cannot Prove:}
\begin{itemize}
    \setlength\itemsep{0em}
    \item Global optimality of solution
    \item Uniqueness of embedding (up to rotation/reflection)
    \item Preservation of specific geometric properties
\end{itemize}

\end{frame}

%==============================================================================
% SLIDE 38: Interactive t-SNE
%==============================================================================
\begin{frame}{Interactive t-SNE: Real-time Exploration}

\textbf{Interactive Tools and Frameworks:}

\vspace{0.2cm}
\textbf{1. TensorBoard Projector}
\begin{itemize}
    \setlength\itemsep{0em}
    \item Web-based visualization
    \item Supports metadata/labels
    \item Real-time parameter adjustment
    \item Integration with TensorFlow
\end{itemize}

\vspace{0.15cm}
\textbf{2. Embedding Projector (standalone)}
\begin{itemize}
    \setlength\itemsep{0em}
    \item \url{projector.tensorflow.org}
    \item Upload custom data
    \item PCA → t-SNE pipeline
    \item 3D visualization option
\end{itemize}

\vspace{0.15cm}
\textbf{3. R Shiny Applications}
\begin{itemize}
    \setlength\itemsep{0em}
    \item Custom dashboards with \texttt{shiny}
    \item Real-time perplexity adjustment
    \item Linked brushing with other plots
    \item Export capabilities
\end{itemize}

\vspace{0.15cm}
\textbf{Key Features for Interactivity:}
\begin{itemize}
    \setlength\itemsep{0em}
    \item Hover tooltips with point information
    \item Zoom and pan capabilities
    \item Dynamic coloring by metadata
    \item Animation of optimization process
\end{itemize}

\end{frame}

%==============================================================================
% SLIDE 39: Advanced R Implementation
%==============================================================================
\begin{frame}{Advanced R Implementation: Beyond Basics}

\textbf{Optimizing t-SNE in R:}

\vspace{0.2cm}
\textbf{1. Parallel Processing}
\begin{itemize}
    \setlength\itemsep{0em}
    \item Use \texttt{parallel} package for multiple runs
    \item \texttt{foreach} with \texttt{doParallel} backend
    \item Speed up 3-4× on multicore systems
\end{itemize}

\vspace{0.15cm}
\textbf{2. Custom Distance Metrics}
\begin{itemize}
    \setlength\itemsep{0em}
    \item \texttt{Rtsne} accepts distance matrices
    \item Use \texttt{proxy::dist()} for custom metrics
    \item Examples: cosine, Manhattan, Mahalanobis
\end{itemize}

\vspace{0.15cm}
\textbf{3. Preprocessing Pipeline}
\begin{itemize}
    \setlength\itemsep{0em}
    \item \texttt{scale()}: Standardization
    \item \texttt{prcomp()}: PCA reduction
    \item \texttt{Rtsne::normalize\_input()}: Built-in normalization
\end{itemize}

\vspace{0.15cm}
\textbf{4. Visualization Enhancements}
\begin{itemize}
    \setlength\itemsep{0em}
    \item \texttt{ggplot2}: Publication-quality plots
    \item \texttt{plotly}: Interactive 3D t-SNE
    \item \texttt{gganimate}: Animate convergence
    \item \texttt{ggrepel}: Non-overlapping labels
\end{itemize}

\vspace{0.1cm}
\begin{exampleblock}{Pro Tip}
\footnotesize
Save intermediate iterations: \texttt{Rtsne(..., Y\_init = prev\_Y)}\\
Allows continuation if interrupted and convergence analysis
\end{exampleblock}

\end{frame}

%==============================================================================
% SLIDE 40: Future Directions and Research
%==============================================================================
\begin{frame}{Future Directions and Current Research}

\textbf{Active Research Areas:}

\vspace{0.2cm}
\textbf{1. Theoretical Foundations}
\begin{itemize}
    \setlength\itemsep{0em}
    \item Formal guarantees for cluster preservation
    \item Connection to spectral methods
    \item Optimal kernel selection theory
\end{itemize}

\vspace{0.15cm}
\textbf{2. Algorithmic Improvements}
\begin{itemize}
    \setlength\itemsep{0em}
    \item GPU acceleration (RAPIDS cuML)
    \item Quantum t-SNE algorithms
    \item Online/streaming t-SNE
    \item Hierarchical t-SNE for multi-scale
\end{itemize}

\vspace{0.15cm}
\textbf{3. Applications and Extensions}
\begin{itemize}
    \setlength\itemsep{0em}
    \item Time-series t-SNE (preserving temporal structure)
    \item Multi-view t-SNE (multiple data modalities)
    \item Supervised t-SNE (using label information)
    \item Uncertainty quantification in embeddings
\end{itemize}

\vspace{0.15cm}
\textbf{4. Integration with Deep Learning}
\begin{itemize}
    \setlength\itemsep{0em}
    \item End-to-end trainable t-SNE layers
    \item Variational t-SNE
    \item Adversarial t-SNE for robust embeddings
\end{itemize}

\end{frame}
\end{document}